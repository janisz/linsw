\documentclass{mini}
\usepackage[utf8]{inputenc}
\usepackage[polish]{babel}
\usepackage{enumitem}
\usepackage{todonotes}
\presetkeys
	{todonotes}
	{inline}{}
	
 \usepackage{listings}
 \lstset{
         basicstyle=\footnotesize\ttfamily, % Standardschrift
         numbers=left,               % Ort der Zeilennummern
         numberstyle=\tiny,          % Stil der Zeilennummern
         %stepnumber=2,               % Abstand zwischen den Zeilennummern
         numbersep=5pt,              % Abstand der Nummern zum Text
         tabsize=2,                  % Groesse von Tabs
         extendedchars=false,         %
         breaklines=true,            % Zeilen werden Umgebrochen
         showspaces=false,           % Leerzeichen anzeigen ?
         showtabs=false,             % Tabs anzeigen ?
         xleftmargin=17pt,
         framexleftmargin=17pt,
         framexrightmargin=5pt,
         framexbottommargin=4pt,
         %backgroundcolor=\color{lightgray},
         showstringspaces=false      % Leerzeichen in Strings anzeigen ?        
 }
    %\DeclareCaptionFont{blue}{\color{blue}} 

  %\captionsetup[lstlisting]{singlelinecheck=false, labelfont={blue}, textfont={blue}}
  \usepackage{caption}
\DeclareCaptionFont{white}{\color{white}}
\DeclareCaptionFormat{listing}{\colorbox[cmyk]{0.43, 0.35, 0.35,0.01}{\parbox{\textwidth}{\hspace{15pt}#1#2#3}}}
\captionsetup[lstlisting]{format=listing,labelfont=white,textfont=white, singlelinecheck=false, margin=0pt, font={bf,footnotesize}}


%------------------------------------------------------------------------------%
\title{Linux w Systemach Wbudowanych}
\author{Tomasz Janiszewski}
\monthyear{\today}
%------------------------------------------------------------------------------%

\usepackage{listings}
\begin{document}
\maketitle
\tableofcontents

\newpage

\section{Opis zadania}
\begin{enumerate}
\item Przygotować aplikację w języku C, obsługującą przyciski i diody LED.
\item Aplikacja powinna reagować na zmiany stanu przycisków bez oczekiwania aktywnego
\item Funkcjonalność aplikacji, może być ustalona przez studenta przykłady
(można wymyślić własną, ciekawszą aplikację): stoper, z LED-ami potwierdzającymi operację i
z możliwością pomiaru „międzyczasów” bądź zamek szyfrowy z logowaniem prób otwarcia. LED-y sygnalizują:
otwarcie zamka, błędny kod, alarm po kilkukrotnym wprowadzeniu błędnego kodu.
\item Przetestować korzystanie z debuggera (gdb) przy uruchamianiu aplikacji
\item Zagadnienie dodatkowe: przygotować równoważną aplikację w wybranym języku skryptowym i porównać
wydajność obu implementacji
\end{enumerate}

\section{Opis rozwiązania}

Aplikacja to gra w kótrej gracz ma powtórzyć wylosowaną sekwencję
świecących się diod. Przy implementacji oparto się o 
bibliotekę \url{https://www.ridgerun.com/developer/wiki/index.php/Gpio-int-test.c}

\subsection{Konfiguracja \emph{buildroota}}
\subsubsection{Potrzebne pakiety}
W celu uruchomienia aplikacji i mozliwości jej debugowania należy skonfigurowac poniższe opcje
\begin{enumerate}
\item python
\item gdb (gdbserver)
\end{enumerate}
\subsubsection{Dodatkowa konfiguracja}
W celu debugowania aplikcaji należy skonfigurowac poniższe opcje
\begin{enumerate}
\item \texttt{Toolchain/Build cross gdb for the host}
\item \texttt{Build options/Build packages with debugging symbols}
\end{enumerate}
\subsubsection{Konfiguracja linuxa}
W celu włączenia obsługi GPIO przez sysfs należy skonfigurować poniższe opcje
\begin{enumerate}
\item \texttt{Device Drivers / GPIO Support / /sys/class/gpio/... (sysfs interface)}
\end{enumerate}


\subsection{Aplikacja w języku \emph{C}}
\lstinputlisting[language=c,caption=main.c]{main.c}
\lstinputlisting[language=make,caption=Makefile]{Makefile}

\subsection{Pakietyzacja aplikacji w środowisku \emph{buildroot}}
Aby zamknac aplikacje w pakiet przygotowano poniższe pliki konfiguracyjne
\lstinputlisting[language=make,caption=package/memo/Config.in]{Config.in}
\lstinputlisting[language=make,caption=package/memo/memo.mk]{memo.mk}
Oraz dodano poniższe linijki do pliku \texttt{package/Config.in}
\begin{lstlisting}
menu "Games" 
        source "package/memo/Config.in" 
endmenu
\end{lstlisting}

\subsection{Debugowanie aplikacji}
Aby debugować aplikację na płytce nalezy uruchomić 
\begin{lstlisting}[language=bash]
gdbserver host:7654 memo 5
\end{lstlisting}
a na hoście
\begin{lstlisting}[language=bash]
output/host/usr/bin/arm-none-linux-gnueabi-gdb /malina/buildroot/output/build/ledkeys-1.0/keys
\end{lstlisting}
\begin{lstlisting}[language=bash]
(gdb) set sysroot output/target
(gdb) target remote xxx.yyy.zzz.vvv:7654
\end{lstlisting}

\subsection{Aplikcja w języku skryptowym}
Aplikacja o równoważnej funckjonalności wjęzyku Python
\lstinputlisting[language=Python,caption=memo.py]{memo.py}

\end{document}
