\documentclass{mini}
\usepackage[utf8]{inputenc}
\usepackage[polish]{babel}
\usepackage{enumitem}
\usepackage{todonotes}
\presetkeys
	{todonotes}
	{inline}{}
	
 \usepackage{listings}
 \lstset{
         basicstyle=\footnotesize\ttfamily, % Standardschrift
         numbers=left,               % Ort der Zeilennummern
         numberstyle=\tiny,          % Stil der Zeilennummern
         %stepnumber=2,               % Abstand zwischen den Zeilennummern
         numbersep=5pt,              % Abstand der Nummern zum Text
         tabsize=2,                  % Groesse von Tabs
         extendedchars=false,         %
         breaklines=true,            % Zeilen werden Umgebrochen
         keywordstyle=\textbf,
            frame=b,         
 %        keywordstyle=[1]\textbf,    % Stil der Keywords
 %        keywordstyle=[2]\textbf,    %
 %        keywordstyle=[3]\textbf,    %
 %        keywordstyle=[4]\textbf,   \sqrt{\sqrt{}} %
         stringstyle=\color{white}\ttfamily, % Farbe der String
         showspaces=false,           % Leerzeichen anzeigen ?
         showtabs=false,             % Tabs anzeigen ?
         xleftmargin=17pt,
         framexleftmargin=17pt,
         framexrightmargin=5pt,
         framexbottommargin=4pt,
         %backgroundcolor=\color{lightgray},
         showstringspaces=false      % Leerzeichen in Strings anzeigen ?        
 }
    %\DeclareCaptionFont{blue}{\color{blue}} 

  %\captionsetup[lstlisting]{singlelinecheck=false, labelfont={blue}, textfont={blue}}
  \usepackage{caption}
\DeclareCaptionFont{white}{\color{white}}
\DeclareCaptionFormat{listing}{\colorbox[cmyk]{0.43, 0.35, 0.35,0.01}{\parbox{\textwidth}{\hspace{15pt}#1#2#3}}}
\captionsetup[lstlisting]{format=listing,labelfont=white,textfont=white, singlelinecheck=false, margin=0pt, font={bf,footnotesize}}


%------------------------------------------------------------------------------%
\title{Linux w Systemach Wbudowanych}
\author{Tomasz Janiszewski}
\monthyear{\today}
%------------------------------------------------------------------------------%

\usepackage{listings}
\begin{document}
\maketitle
\tableofcontents

\newpage

\section{Opis zadania}
Przygotować obraz systemu Linux, używający ,,initramfs'' jako głównego systemu plików.
\begin{itemize}
\item RPi powinno automatycznie podłączyć się do sieci, używając DHCP do otrzymania parametrów sieci
\item System powinien mieć ustaloną nazwę jako: ,,imię nazwisko'' autora
\item Obraz systemu powinien zawierać dodatkowe programy wskazane przez prowadzącego (powiedzmy, że na przykład chcemy, żeby był tam prosty serwer WWW udostępniający statyczne strony HTML)
\item System powinien synchronizować czas systemowy z serwerem czasu
\item Hasło użytkownika ,,root'' oraz hasło użytkownika domyślnego ,,default'' powinno być odpowiednio ustawione przy starcie systemu.
\end{itemize}

\section{Opis rozwiązania}

\subsection{Konfiguracja \emph{buildroot}}
\begin{enumerate}
	\item Sklonowanie repozytorium \emph{buildroot}
	\begin{lstlisting}[language=bash]
	git clone git://git.buildroot.net/buildroot --depth 1
	\end{lstlisting}
	\item Przechodzimy do utworzonego katalogu
	\begin{lstlisting}[language=bash]
	cd buildroot
	\end{lstlisting}	
	\item Ustawiamy konfigurację dla Raspberry Pi
	\begin{lstlisting}[language=bash]
	make raspberrypi_defconfig
	\end{lstlisting}	
	\item Wypełniamy plik definiujący konto użytkownika
	\begin{lstlisting}[language=bash]
	echo "default -1 default -1 =password - /bin/sh - " > user.tables
	\end{lstlisting}		
	\item Przy pomocy \emph{make menuconfig}
	konfigurujemy poniższe ustawienia\\
	\begin{itemize}	
		\item System configuration
		\begin{itemize}
			\item System hostname
			\item Root password
			\item Network interface to configure through DHCP
			\item Install timezone info
			\item Path to the users tables
		\end{itemize}
		\item Filesystem images
		\begin{itemize}
			\item initial RAM filesystem linked into linux kernel
		\end{itemize}
		\item Packages 
		\begin{itemize}
			\item ntp
			\item dhcp
			\item lighttpd
		\end{itemize}
	\end{itemize}
	\item Budujemy system
	\begin{lstlisting}[language=bash]
	make
	\end{lstlisting}	
	\item Ustawiamy plik dla serwera WWW
	\begin{lstlisting}[language=bash]
	echo "It is working" > output/target/var/www/index.html
	make 
	\end{lstlisting}	
\end{enumerate}

\subsection{Uruchomienie systemu na płytce}
W celu uruchomiena sustemu na płytce, należy skopiować pliki z katalogu images na kartę pamięci używając poniższego polecenia:
	\begin{lstlisting}[language=bash]
	cp output/images/rpi-firmware/* <katalog karty SD>
	cp output/images/zImage <katalog karty SD>
	\end{lstlisting}	
	
\subsection{Monitorowanie pracy użądzenia}
Do monitorowania pracy użądzenia użyto narzędzia \emph{minicom}. Po restarcie urządzenia może okazać się konieczne ręczne urochomienie konfiguracji DHCP przez \emph{udhcpc}




\end{document}
