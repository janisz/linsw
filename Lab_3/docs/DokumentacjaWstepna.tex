\documentclass{mini}
\usepackage[utf8]{inputenc}
\usepackage[polish]{babel}
\usepackage{enumitem}
\usepackage{todonotes}
\presetkeys
	{todonotes}
	{inline}{}
	
 \usepackage{listings}
 \lstset{
         basicstyle=\footnotesize\ttfamily, % Standardschrift
         numbers=left,               % Ort der Zeilennummern
         numberstyle=\tiny,          % Stil der Zeilennummern
         %stepnumber=2,               % Abstand zwischen den Zeilennummern
         numbersep=5pt,              % Abstand der Nummern zum Text
         tabsize=2,                  % Groesse von Tabs
         extendedchars=false,         %
         breaklines=true,            % Zeilen werden Umgebrochen
         showspaces=false,           % Leerzeichen anzeigen ?
         showtabs=false,             % Tabs anzeigen ?
         xleftmargin=17pt,
         framexleftmargin=17pt,
         framexrightmargin=5pt,
         framexbottommargin=4pt,
         %backgroundcolor=\color{lightgray},
         showstringspaces=false      % Leerzeichen in Strings anzeigen ?        
 }
    %\DeclareCaptionFont{blue}{\color{blue}} 

  %\captionsetup[lstlisting]{singlelinecheck=false, labelfont={blue}, textfont={blue}}
  \usepackage{caption}
\DeclareCaptionFont{white}{\color{white}}
\DeclareCaptionFormat{listing}{\colorbox[cmyk]{0.43, 0.35, 0.35,0.01}{\parbox{\textwidth}{\hspace{15pt}#1#2#3}}}
\captionsetup[lstlisting]{format=listing,labelfont=white,textfont=white, singlelinecheck=false, margin=0pt, font={bf,footnotesize}}


%------------------------------------------------------------------------------%
\title{Linux w Systemach Wbudowanych}
\author{Tomasz Janiszewski}
\monthyear{\today}
%------------------------------------------------------------------------------%

\usepackage{listings}
\begin{document}
\maketitle
\tableofcontents

\newpage

\section{Opis zadania}
\begin{itemize}
\item Przygotować ,,administracyjny'' system Linux pracujący w 
initramfs, umożliwiający przygotowanie karty pamięci SD do 
instalacji systemu Linux pracującego z systemem plików 
e2fs, montowanym z partycji 2 na karcie SD

\item Przygotować ,,użytkowy'' system Linux pracujący z 
systemem plików e2fs, zawierający serwer WWW, 
udostępniający pliki z partycji 3 na karcie SD i umożliwiający 
wgrywanie nowych plików po podaniu hasła.

\item Przygotować bootloader, umożliwiający 
określenie (przy pomocy przycisku), który system ma zostać 
załadowany
\end{itemize}

\section{Opis rozwiązania}
Jako systemu ,,administracyjnego'' użyto rozwiązania z zadania 1.
System do wgrywania plików został oparty o klasę \emph{SimpleHTTPServer}\footnote{https://docs.python.org/2/library/simplehttpserver.html},
która jest standardową klasą języka Python. Do uruchamiania aplikacji, zarządzania
jej logami oraz pilnowaniem aby była ciągle uruchomiona użyto \emph{supervisor}\footnote{http://supervisord.org/}.
Jako bootloadera użyto \emph{barebox}\footnote{http://www.barebox.org/}.
Po uruchomieniu oczekuje on 3 sekundy, na
naciśnięcie przycisku. Jeśli w tym czasie przycisk nie został naciśnięty to
uruchamia sie ,,użytkowy'' system operacyjny, w przciwnym przypadku
ładuje się system ,,administracyjny''.
Do notyfikacji o tym że układ znajduje się w fazie booloadera informuje dioda ACT
płytk. Sterowanie diodą odbywa się za pomocą wbudowanego oprogramowania \emph{trigger}\footnote{http://wiki.barebox.org/doku.php?id=commands:trigger}
wspieranego w \emph{Raspberry Pi} od wersji \emph{2015.04.0}\footnote{http://www.spinics.net/lists/u-boot-v2/msg21779.html}.

System ,,użytkowy'' powstał w wyniku następujących zmian w konfiguracji systemu administracyjnego:
\begin{enumerate}
	\item Sklonowanie repozytorium \emph{buildroot} i przejście na stabliną wersję
	\item Przy pomocy \emph{make menuconfig}
	konfigurujemy poniższe ustawienia\\
	\begin{itemize}	
		\item System configuration $\rightarrow$ overlays \emph{overlays}
		\item Filesystem images
		\begin{itemize}
			\item initial RAM filesystem linked into linux kernel (odznaczamy)
			\item ext2/3/4 root filesystem
		\end{itemize}
		\item Packages 
		\begin{itemize}
			\item python
			\item supervisor
		\end{itemize}
		\item Bootloaders
		\begin{itemize}
			\item Barebox
			\begin{itemize}
				\item version \emph{2015.04.0}
				\item boar defconfig \emph{rpi}
			\end{itemize}
		\end{itemize}
	\end{itemize}
	\item Przy pomocoy \emph{make barebox-menuconfig}
	konifugurujemy poniższe ustawienia
	\begin{itemize}
		\item Drivers $\rightarrow$ LED Support $\rightarrow$ 
		\begin{itemize}
			\item gpio LED support
			\item LED triggers support
		\end{itemize}
		\item Commands
		\begin{itemize}
			\item Boot $\rightarrow$ bootz
			\item Hardware manipulation $\rightarrow$ trigger
		\end{itemize}
	\end{itemize}
	\item Tworzymy następującą strukturę katalogów

	\begin{lstlisting}[language=bash]
mkdir -p overlays/etc/init.d
mkdir -p var/www
	\end{lstlisting}	
	\lstinputlisting[caption=overlays/etc/supervisord.conf]{../overlays/etc/supervisord.conf}
	\lstinputlisting[language=Python,caption=overlays/var/www/SimpleHTTPServerWithUpload.py]{../overlays/var/www/SimpleHTTPServerWithUpload.py}	
	\item Budujemy system
	\begin{lstlisting}[language=bash]
	make
	\end{lstlisting}	
\end{enumerate}

\subsection{Konfiguracja środowiska barebox}
Po uruchomieniu bootloadera należy użyć polecenia \emph{edit} i \emph{saveenv} aby
zapisać ustawienia.
\lstinputlisting[caption=/boot/bin/init]{../init}

\end{document}
